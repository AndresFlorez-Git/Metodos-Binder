\documentclass{article} %tipo de documento
\usepackage[utf8]{inputenc}
\usepackage{subfigure}  %subfiguras
\usepackage{float}
\usepackage[spanish]{babel}
\usepackage{graphicx}
\usepackage{amsmath}
\usepackage{endnotes}
\date{\today}
\usepackage{natbib}
\usepackage{fancyhdr}
\usepackage{color}
\usepackage{graphicx} %gráficas
\setlength{\textwidth}{170mm}
\setlength{\textheight}{230mm}
\setlength{\topmargin}{-15mm}
\setlength{\oddsidemargin}{-3mm}
\setlength{\evensidemargin}{30mm}
\title{Resultados Tarea 2 Métodos Computacionales}
\author{Andrés Felipe Flórez Sierra - af.florez12@uniandes.edu.co\\Universidad de los Andes}

\date{July 2019}

\begin{document}

\maketitle

\section{Introduction}
En el presente documento se exponen los resultados obtenidos en la realización de la tarea 2 de Métodos computacionales.
\section{Fourier en manejo de imágenes}
Se presentan los resultados de la creación de una imagen híbrida a partir de dos fotografías distintas.
\subsection{Proceso}\label{sub:proces_fourier}
Para la creación de la imagen hibrida utilizando el lenguaje de programación Python 3.7 es la siguiente:
\begin{itemize}
    \item Se importan las las fotografías, en este caso; cara\_02\_risesMF.png y de cara\_03\_grisesMF.png.
    \item Se obtiene el espectro de fourier de cada una de las imágenes.
    \item Se obtiene las frecuencias asociadas a las amplitudes del espectro de fourier.En este punto es necesario identificar la información frecuencial que se desee filtrar por el filtro gaussiano, que en este caso se puede aproximar a la implementación de dos filtros; Pasa-Altas y Pasa-Bajas.
    \item Se aplican los filtros respectivos a cada una de las imágenes dependiendo de la información identificada en las frecuencias asociadas a las amplitudes del espectro de fourier.
    \item Se visualiza el espectro de fourier para verificar la aplicación del filtro.
    \item Se combina la información de las amplitudes del espectro de fourier en una sola imagen, la cual sera el resultado.
    \item para visualizar la imagen híbrida, se aplica la transformada inversa de fourier.
\end{itemize}

\subsection{Resultado}
Los resultados presentados en la figura \ref{fig:proces_fourier}  cumplen con el orden expuesto en la sección  (\ref{sub:proces_fourier}).
\begin{figure}[H]
    \includegraphics[height=23cm,width=20cm]{FFT2D.png}
    \caption{Proceso para obtener la imagen híbrida }
    \label{fig:proces_fourier}
\end{figure}
\section{ODEs, movimiento orbital}

%%%%%%%%%%%%%%%%%%%%%%%%%%%%%%%%%%%%%%%%
En la figura \ref{fig:Posicion_Orbital} se evidencia que los métodos LeapFrog y Runge-Kutta son los más efectivos para calcular las posiciones orbitales. El método de Euler presenta resultados de baja con fiabilidad para dt bajos.
\begin{figure}[H]
    \centering
    \includegraphics[height=20cm,width=20cm]{Posicion_Orbital.png}
    \caption{Posición Orbital}
    \label{fig:Posicion_Orbital}
\end{figure}
%%%%%%%%%%%%%%%%%%%%%%%%%%%%%%%%%%%%%%%%
Se evidencia en la figura \ref{fig:Velocidad_Orbital}, que los métodos LeapFrog y Runge-Kutta son los más efectivos para calcular las velocidades orbitales. El método de Euler presenta resultados de baja con fiabilidad para dt bajos. El resultado se degrada entre mas tiempo pasa.
\begin{figure}[H]
    \centering
    \includegraphics[height=20cm,width=20cm]{Velocidad_Orbital.png}
    \caption{Velocidad Orbital}
    \label{fig:Velocidad_Orbital}
\end{figure}
%%%%%%%%%%%%%%%%%%%%%%%%%%%%%%%%%%%%%%%%
Se evidencia en la figura \ref{fig:Momentum_Angular} que los métodos de LeapFrog y Runge-Kutta conservan el momento angular, sin embargo el método de Euler no.
\begin{figure}[H]
    \centering
    \includegraphics[height=20cm,width=20cm]{Momentum_Angular.png}
    \caption{Momentum Angular}
    \label{fig:Momentum_Angular}
\end{figure}
%%%%%%%%%%%%%%%%%%%%%%%%%%%%%%%%%%%%%%%%
Se evidencia en la figura \ref{fig:Energia_Euler} el análisis energético del método de Euler, dando como resultado que la energía no se conserva. 
\begin{figure}[H]
    \centering
    \includegraphics[height=20cm,width=20cm]{Energia_Euler.png}
    \caption{Energía Euler}
    \label{fig:Energia_Euler}
\end{figure}
%%%%%%%%%%%%%%%%%%%%%%%%%%%%%%%%%%%%%%%%
Se evidencia en la figura \ref{fig:Energia_LeapFrog} el análisis energético del método de LeapFrog, dando como resultado que la energía se conserva. 
\begin{figure}[H]
    \centering
    \includegraphics[height=20cm,width=20cm]{Energia_LeapFrog.png}
    \caption{Energia LeapFrog}
    \label{fig:Energia_LeapFrog}
\end{figure}
%%%%%%%%%%%%%%%%%%%%%%%%%%%%%%%%%%%%%%%%
Se evidencia en la figura \ref{fig:Energia_RK} el análisis energético del método de Runge-Kutta, dando como resultado que la energía No se conserva. 
\begin{figure}[H]
    \centering
    \includegraphics[height=20cm,width=20cm]{Energia_RK.png}
    \caption{Energia Runge-Kutta}
    \label{fig:Energia_RK}
\end{figure}


\end{document}
